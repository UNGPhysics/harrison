\documentclass[11pt]{article}
\usepackage[margin=0.75in]{geometry}
\usepackage{graphicx}
\usepackage{multicol}
\usepackage{float}
\usepackage{upgreek}
\usepackage{amsmath}
\usepackage{scrextend}

\setlength{\parindent}{0mm}

\begin{document}

{\centering
\textbf{Topics 4-5 Study Guide} \par
\vspace{\baselineskip}
}

\textbf{I. Things to memorize.}
The first question on the test will be a blank space where you will be asked to reproduce the following information from memory.
\vspace{0.5\baselineskip}

\begin{addmargin}[1em]{0em}% 1em left, 0em right
\textbf{A. Newton's Laws of Motion} \\
$\bullet$ A body at rest remains at rest or, if in motion, remains in motion at constant velocity unless acted on by a net external force. \\
$\bullet$ The acceleration of a system is directly proportional to and in the same direction as the net external force acting on the system and is inversely proportion to its mass ($\Sigma \vec{F} = m \vec{a}$). \\
$\bullet$ Whenever one body exerts a force on a second body, the first body experiences a force that is equal in magnitude and opposite in direction to the force that it exerts. \\

\textbf{B. Units:} \\
The SI unit of force is the Newton ($N$) which is equivalent to a $kg\cdot m/s^2$. \\

\textbf{C. Rules for Common Forces:} \\
$\bullet$ Gravity. Magnitude: Mg. Direction: straight down. \\
$\bullet$ Normal Force. Magnitude: problem dependent. Direction: perpendicular to the surface. \\
$\bullet$ Static Friction. Magnitude: $f_s \leq \mu_s N$. Direction: parallel to the surface, resists motion. \\
$\bullet$ Kinetic Friction. Magnitude: $f_k = \mu_k N$. Direction: parallel to the surface, resists motion. \\
$\bullet$ Tension. Magnitude: problem dependent. Direction: parallel to the rope. \\
$\bullet$ Spring Force. $\vec{F} = -k\vec{x}$. \\
\end{addmargin}

\vspace{\baselineskip}
\textbf{II. Proofs.}
Show that the acceleration of a block sliding down a ramp inclined at angle $\theta$ is given by $g\sin\theta - \mu_k g\cos\theta$.
Do this by first drawing a precise free-body diagram with all forces and components clearly shown and labeled.

\vspace{\baselineskip}
\textbf{III. Problem solving.}
There will be 1 or 2 questions directly from the HW and 1 or 2 original questions.

\end{document}
