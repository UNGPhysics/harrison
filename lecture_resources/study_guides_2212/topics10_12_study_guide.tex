\documentclass[11pt]{article}
\usepackage[margin=0.75in]{geometry}
\usepackage{graphicx}
\usepackage{multicol}
\usepackage{float}
\usepackage{upgreek}
\usepackage{amsmath}
\usepackage{scrextend}

\setlength{\parindent}{0mm}

\begin{document}

{\centering
\textbf{Topics 10-12 Study Guide} \par
\vspace{\baselineskip}
}

\textbf{I. Things to memorize.}
The first question on the test will be a blank space where you will be asked to reproduce the following information from memory.
\vspace{0.5\baselineskip}

\begin{addmargin}[1em]{0em}% 1em left, 0em right
\textbf{A. Units:}

\vspace{0.25\baselineskip}
\begin{tabular}{ |c|c| } 
\hline
Quantity & SI Unit \\ 
\hline
Magnetic field (B) & Tesla (T) or N$\cdot$s/(C$\cdot$m) or kg/(C$\cdot$s) \\ 
\hline
\end{tabular}
\vspace{0.75\baselineskip}

\textbf{B. Equations for electrodynamics} \\
$\bullet$ Lorentz Force: $F = q (\vec{v} \times \vec{B} + \vec{E})$ \\
$\bullet$ Biot-Savart Law: $d\vec{B} = \frac{\mu_0}{4\pi} \frac{I d\vec{l} \times \hat{r}}{r^2}$ \\

\textbf{C. Maxwell's equations} \\
$\bullet$ Gauss's Law: $\oint \vec{E} \cdot d\vec{A} = q_{enc}/\varepsilon_0$ \\
$\bullet$ Gauss's Law for Magnetism: $\oint \vec{B} \cdot d\vec{A} = 0$ \\
$\bullet$ Ampere-Maxwell Law: $\oint \vec{B} \cdot d\vec{s} = \mu_0 I + \epsilon_0 \mu_0 \frac{d\Phi_E}{dt}$ \\
$\bullet$ Faraday's Law: $\oint \vec{E} \cdot d\vec{s} = - \frac{d \Phi_m}{dt}$ \\

\end{addmargin}

\vspace{\baselineskip}
\textbf{II. Proofs.} \\
$\bullet$ Show that the magnetic field due to an infinitely long straight wire is given by $B = \frac{\mu_0 I}{2 \pi R}$ using either Ampere's Law or the  Biot-Savart Law. \\

\vspace{\baselineskip}
\textbf{III. Problem solving.} \\
There will be 1 or 2 questions directly from the HW and 1 or 2 original questions.

\end{document}
