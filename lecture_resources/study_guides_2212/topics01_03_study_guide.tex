\documentclass[11pt]{article}
\usepackage[margin=0.75in]{geometry}
\usepackage{graphicx}
\usepackage{multicol}
\usepackage{float}
\usepackage{upgreek}
\usepackage{amsmath}
\usepackage{scrextend}

\setlength{\parindent}{0mm}

\begin{document}

{\centering
\textbf{Topics 1-3 Study Guide} \par
\vspace{\baselineskip}
}

\textbf{I. Things to memorize.}
The first question on the test will be a blank space where you will be asked to reproduce the following information from memory.
\vspace{0.5\baselineskip}

\begin{addmargin}[1em]{0em}% 1em left, 0em right
\textbf{A. Units:}

\vspace{0.25\baselineskip}
\begin{tabular}{ |c|c| } 
\hline
Quantity & SI Unit \\ 
\hline
frequency (f) & Hz or $s^{-1}$ \\ 
\hline
period (T) & seconds (s) \\ 
\hline
angular frequency ($\omega$) & rad/s \\ 
\hline
wavelength ($\lambda$) & meters (m) \\ 
\hline
wave number (k) & rad/m \\ 
\hline
spring constant ($k_s$) & N/m \\ 
\hline
\end{tabular}
\vspace{0.75\baselineskip}

\textbf{B. Equations for general simple harmonic motion (SHM)} \\
$\bullet$ $f = \frac{1}{T}$ \\
$\bullet$ $\omega = 2\pi f$ \\

\textbf{C. Equations for a spring-mass oscillator} \\
$\bullet$ $\omega = \sqrt{\frac{k_s}{m}}$ \\

\textbf{D. Equations for a simple pendulum} \\
$\bullet$ $\omega = \sqrt{\frac{g}{L}}$ \\

\textbf{E. Equations for sinusoidal waves} \\
$\bullet$ $y(x,t) = A\sin(kx \pm \omega t + \phi)$ \\
$\bullet$ $k = \frac{2\pi}{\lambda}$ \\
\end{addmargin}

\vspace{\baselineskip}
\textbf{II. Conceptual questions.} \\
$\bullet$ Be able to switch back and forth between $f$, $T$, and $\omega$ for any of the systems we've studied. \\
$\bullet$ Demonsrate an understanding of the physical meaning to all of the variables in the equation $y(x,t) = A\sin(kx \pm \omega t + \phi)$. \\

\vspace{\baselineskip}
\textbf{III. Problem solving.} \\
There will be 1 or 2 questions directly from the HW and 1 or 2 original questions.
Topics will include SHM (spring/mass and pendulum), the mathematics of waves, wave interference, and the doppler effect.

\end{document}
