\documentclass[11pt]{article}
\usepackage[margin=0.75in]{geometry}
\usepackage{graphicx}
\usepackage{multicol}
\usepackage{float}
\usepackage{upgreek}
\usepackage{amsmath}
\usepackage{scrextend}

\setlength{\parindent}{0mm}

\begin{document}

{\centering
\textbf{Topic 1 Study Guide} \par
\vspace{\baselineskip}
}

\textbf{I. Things to memorize.}
The first question on the test will be a blank space where you will be asked to reproduce the following information from memory.
\vspace{0.5\baselineskip}

\begin{addmargin}[1em]{0em}% 1em left, 0em right
\textbf{A. Units:}

\vspace{0.25\baselineskip}
\begin{tabular}{ |c|c| } 
\hline
Quantity & SI Unit \\ 
\hline
distance (x) & meters (m) \\ 
\hline
time (t) & seconds (s) \\ 
\hline
velocity (v) & m/s \\ 
\hline
acceleration (a) & $\text{m/s}^2$ \\ 
\hline
\end{tabular}
\vspace{0.75\baselineskip}

\textbf{B. Definitions:} \\
$\bullet$ Velocity - the rate at which a given distance is increasing (or decreasing) \\
\null\quad $\bullet$ Average velocity $v = \frac{\Delta x}{\Delta t}$ \\
\null\quad $\bullet$ Instantaneous velocity $v = \frac{d x}{d t}$ \\
$\bullet$ Acceleration - the rate at which velocity is increasing (or decreasing). $a = \frac{\Delta v}{\Delta t}$. \\
e.g. $v_i = 12 \ m/s$, $a = 5 \ m/s^2$

\vspace{0.25\baselineskip}
\begin{tabular}{ |c|c|c|c|c|c| } 
\hline
time (s) & 0 & 1 & 2 & 3 & 4 \\ 
\hline
v (m/s) & 12 & 17 & 22 & 27 & 32 \\ 
\hline
\end{tabular}
\vspace{0.75\baselineskip}

\textbf{C. Equations:} \\
The distance an object will travel under constant acceleration is given by $\Delta x = v_i t + \frac{1}{2} a t^2$.
\end{addmargin}

\vspace{\baselineskip}
\textbf{II. Proofs.}
The second question on the test will ask you to prove the result $\Delta x = v_i t + \frac{1}{2} a t^2$.
There is more than one valid answer here - see the lecture.

\vspace{\baselineskip}
\textbf{III. Problem solving.}
There will be 1 or 2 questions directly from the HW and 1 or 2 original questions.

\end{document}
